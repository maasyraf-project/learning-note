\lesson{1}{Monday 11 Des 2023 18:20}{Paper Review : Overview of Gammatone Filterbank, based on Hohmann (2002)}

\subsection{Important notes}
\begin{itemize}
	\item The Gammatone filters are commonly used for modelling peripheral filtering in the cochlea
	
	\item On its implementation, there are several approximation that can be considered on modelling the filters.
	
	\begin{enumerate}
		\item Impulse-invariant or frequency-invariant
		\item Multi-zero, one-zero, or all-pole design
		\item Complex or real filter output
	\end{enumerate}
	
	\item Several implementation of Gammatone filterbanks are simulate peripheral filtering stage on computational auditory model; speech- and music-coder; and used as a block process in speech recognition task (Hahmann, 2002).
	
\end{itemize}

\subsection{Technical implementation: Synthesis (based on Hohmann, 2002)}
In this model, the Gammatone filterbank has computationally efficient implementation, which can be categorized as impulse-invariant, all-pole design, and result in complex values output. 

First of all, consider a sampled of complex Gammatone impulse response:

\begin{equation}
	g_\gamma (n) = n^{\gamma - 1} \cdot \tilde{a}^n , n>=0
\end{equation}

\begin{equation}
	\tilde{a} = \lambda \cdot exp(\imath\beta)
\end{equation}

On above equations, 
\begin{itemize}
	\item $\gamma$ : filter order
	\item \(a\) : complex analog Gammatone impulse response
	\item $\lambda$ : damping bandwidth parameter
	\item \(n\) : index of impulse response sample
\end{itemize}

Then, on \(z-\)domain, the frequency response of first-order complex bandpass filter can be expressed as,

\begin{equation}
	G_1(z) = \frac{1}{1-\tilde{a}z^{-1}}
\end{equation}

On the higher order of filter design, the model then can be defnied by take derivation on $\zeta$-domain of frequency response. Hence,

\begin{equation}
	\zeta[n \cdot g_{\gamma}(n)] = -z \frac{dG(z)}{dz}
\end{equation}

Finally, for 4-th order Gammatone filterbank, we got

\begin{equation}
	G_4(z) = \frac{\tilde{a}z^{-1} + 4 \cdot (\tilde{a}z^{-1})^2 + (\tilde{a}z^{-1})^3}{(1-\tilde{a}z^{-1})^4}
\end{equation}

The main idea of this implementation is to simplify the filter model by omitting the numerator on the frequency response equation above. Then, we obtain a casade version of first-order filter, as shown below 

\begin{equation}
	K_4(z) = \frac{1}{(1-\tilde{a}z^(-1))^4}
\end{equation} 

As note, omitting the numerator has small effect on filter design, sinze poles are more influencing than zeros. 

\begin{equation}
	content...
\end{equation}