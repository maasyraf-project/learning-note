% basic
\usepackage[utf8]{inputenc}
\usepackage[T1]{fontenc}
\usepackage{url}
\usepackage{graphicx}
\usepackage{float}
\usepackage{enumitem}
\usepackage{emptypage}
\usepackage{multicol}
\usepackage{datetime}

\usepackage{amsmath, amsfonts, mathtools, amsthm, amssymb}
\usepackage{mathrsfs}
\usepackage{cancel}
\usepackage{bm}

% theorems
\makeatother
\usepackage{thmtools}
\usepackage[framemethod=TikZ]{mdframed}

\theoremstyle{definition}

\declaretheoremstyle[
headfont=\bfseries\sffamily\color{ForestGreen!70!black}, bodyfont=\normalfont,
mdframed={
	linewidth=2pt,
	rightline=false, topline=false, bottomline=false,
	linecolor=ForestGreen, backgroundcolor=ForestGreen!5,
}
]{thmgreenbox}

\declaretheoremstyle[
headfont=\bfseries\sffamily\color{NavyBlue!70!black}, bodyfont=\normalfont,
mdframed={
	linewidth=2pt,
	rightline=false, topline=false, bottomline=false,
	linecolor=NavyBlue, backgroundcolor=NavyBlue!5,
}
]{thmbluebox}

\declaretheoremstyle[
headfont=\bfseries\sffamily\color{NavyBlue!70!black}, bodyfont=\normalfont,
mdframed={
	linewidth=2pt,
	rightline=false, topline=false, bottomline=false,
	linecolor=NavyBlue
}
]{thmblueline}

\declaretheoremstyle[
headfont=\bfseries\sffamily\color{RawSienna!70!black}, bodyfont=\normalfont,
mdframed={
	linewidth=2pt,
	rightline=false, topline=false, bottomline=false,
	linecolor=RawSienna, backgroundcolor=RawSienna!5,
}
]{thmredbox}

\declaretheoremstyle[
headfont=\bfseries\sffamily\color{RawSienna!70!black}, bodyfont=\normalfont,
numbered=no,
mdframed={
	linewidth=2pt,
	rightline=false, topline=false, bottomline=false,
	linecolor=RawSienna, backgroundcolor=RawSienna!1,
},
qed=\qedsymbol
]{thmproofbox}

\declaretheoremstyle[
headfont=\bfseries\sffamily\color{NavyBlue!70!black}, bodyfont=\normalfont,
numbered=no,
mdframed={
	linewidth=2pt,
	rightline=false, topline=false, bottomline=false,
	linecolor=NavyBlue, backgroundcolor=NavyBlue!1,
},
]{thmexplanationbox}

\declaretheorem[style=thmgreenbox, name=Definition]{definition}
\declaretheorem[style=thmbluebox, numbered=no, name=Example]{eg}
\declaretheorem[style=thmredbox, name=Proposition]{prop}
\declaretheorem[style=thmredbox, name=Theorem]{theorem}
\declaretheorem[style=thmredbox, name=Lemma]{lemma}
\declaretheorem[style=thmredbox, numbered=no, name=Corollary]{corollary}

\declaretheorem[style=thmproofbox, name=Proof]{replacementproof}
\renewenvironment{proof}[1][\proofname]{\vspace{-10pt}\begin{replacementproof}}{\end{replacementproof}}


\declaretheorem[style=thmexplanationbox, name=Proof]{tmpexplanation}
\newenvironment{explanation}[1][]{\vspace{-10pt}\begin{tmpexplanation}}{\end{tmpexplanation}}

\declaretheorem[style=thmblueline, numbered=no, name=Remark]{remark}
\declaretheorem[style=thmblueline, numbered=no, name=Note]{note}

\newtheorem*{uovt}{UOVT}
\newtheorem*{notation}{Notation}
\newtheorem*{previouslyseen}{As previously seen}
\newtheorem*{problem}{Problem}
\newtheorem*{observe}{Observe}
\newtheorem*{property}{Property}
\newtheorem*{intuition}{Intuition}

\usepackage{xifthen}

\def\testdateparts#1{\dateparts#1\relax}
\def\dateparts#1 #2 #3 #4 #5\relax{
	\marginpar{\small\textsf{\mbox{#1 #2 #3 #5}}}
}

\def\@lesson{}%
\newcommand{\lesson}[3]{
	\ifthenelse{\isempty{#3}}{%
		\def\@lesson{Lecture #1}%
	}{%
		\def\@lesson{Lecture #1: #3}%
	}%
	\subsection*{\@lesson}
	\testdateparts{#2}
}